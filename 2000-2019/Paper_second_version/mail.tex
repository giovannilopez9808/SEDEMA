\documentclass[12pt]{article}
\usepackage{fullpage}
\usepackage{setspace}
\usepackage{parskip}
\usepackage{titlesec}
\usepackage[section]{placeins}
\usepackage{xcolor}
\usepackage{breakcites}
\usepackage{lineno}
\usepackage{hyphenat}
\setlength\columnsep{25pt}
\PassOptionsToPackage{hyphens}{url}
\usepackage[colorlinks = true,
            linkcolor = blue,
            urlcolor  = blue,
            citecolor = blue,
            anchorcolor = blue]{hyperref}
\usepackage{etoolbox}
\usepackage{authblk}
\usepackage{graphicx}
\usepackage[space]{grffile}
\usepackage{latexsym}
\usepackage{textcomp}
\usepackage{longtable}
\usepackage{tabulary}
\usepackage{booktabs,array,multirow}
\usepackage{amsfonts,amsmath,amssymb}
\usepackage[utf8]{inputenc}
\usepackage[ngerman,greek,english]{babel}
\pagestyle{empty}
\begin{document}
\begin{minipage}{0.5\linewidth}
    Dear Editor,
\end{minipage}
\begin{minipage}{0.5\linewidth}
    \begin{flushright}
        \today
    \end{flushright}
\end{minipage}

We are submitting for your consideration the revised manuscript, "The Ultraviolet Environment of a Tropical Megacity in Transition: Mexico City 2000-2019" (with ID 2020-08515t).  We wish to thank you for extending the revision deadline, allowing us to greatly improve the manuscript.  We also wish to thank the three reviewers who provided us with very helpful suggestions.  All three reviewers recognized the importance of the work and seemed supportive of publication, if we could successfully address their comments.  The main issues identified by the reviewers were, in our view:
\begin{enumerate}
    \item Need better documentation for the calibration and long term stability of the UV monitoring instruments.  We now describe the calibration procedure in much more detail, evidencing why we are confident in the data.
    \item Need better formulation and description of the TUV model simulations, particularly on the assumed vertical distribution of the pollutants (esp. ozone and aerosols).  We have now simplified the model's vertical structure, described it more precisely, and performed a sensitivity test to ensure that the exact vertical distribution is rather unimportant to the UV Index as long as the total column optical depth is known.
    \item Need a better quantitative summary.  We have now added a table (new Table 3) summarizing the UV Index observed by the RAMA network, inferred from OMI satellite data, and estimated with the TUV model.  The TUV model is in excellent agreement with the RAMA observations, but satellites cannot resolve the effects of intense local pollution on the UV index.
\end{enumerate}
Many other minor changes have been made to the manuscript, in response to the reviewers' comments as well as our own efforts to be more clear and concise.  The numerical values have also changed slightly from the original draft, due to addition of RAMA data from late 2019 (which became available only recently), and due to the changes in the model set up.  These changes are "in the noise" and the conclusions of our study remain unchanged.

Please let us know if we can help with the next steps in any way.  We look forward to hearing from you.

Sincerely,

Adriana Ipiña Hernandez, Ph.D.,

also on behalf of all co-authors.
\end{document}

